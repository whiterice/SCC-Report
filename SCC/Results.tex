%Arc Flash Study Results & Recommendations

\section{Results}
\label{af:results}

\subsection{Results and Recommendations}
\label{af:results:afrr}

\noindent\emph{The Short Circuit Current Device Evaluation Study has shown that:}
\begin{enumerate}
	\item All protective and power distribution equipment within the scope of this study should be sized to withstand the calculated fault duties with the present upstream supply configuration.

	\item All protective devices within the scope of this study should be able to successfully interrupt the faults at levels they could be subjected to.

	\item Any new devices added into the system should have a minimum interrupting current rating as indicated on the short circuit study results page.
\end{enumerate}

\pagebreak

\subsection{Short Circuit Study Details}
\label{af:results:sccd}

\noindent Refer to the Short Circuit Study Observation Section for detailed results.  The short circuit calculations indicate that all equipment for which the Short Circuit withstand data was available, should be able to safely withstand and/or interrupt the fault current available at its terminals.

\pagebreak

\subsection{Study Recommendations}
\label{af:results:afsr}

Based on the studies performed, we recommend the following:
\begin{itemize}
	\item All suggested settings and device parameters shown in the “EQUIPMENT AND PROTECTIVE DEVICE SETTING TABLE” regardless of scope be implemented to achieve improved protection and protective device selectivity.
	\item We also recommend that all electrical equipment throughout your power distribution system that was not a part of this study, be also verified against the available short circuit currents (see Short Circuit Calculation Tables and Single Line Diagram). All equipment should be rated to interrupt/withstand respective
available fault current indicated in the table.
%	\item All modifications and recommendations in the enclosed “Recommended Changes Summary” list should be reviewed and implemented as necessary.
\end{itemize}

\vspace{10mm}
\noindent Thank you for this opportunity to be of service to you.  If you have any questions regarding the recommendations in this report or any other matter, please contact our London Engineering Services office at (519) 474-1175. \newline
\vspace{5mm}
\\
\noindent Sincerely,\newline

\vspace{5mm}
\noindent\textbf{PowerCore Engineering}\newline

%Scott Signature
%\begin{comment}
\begin{multicols}{2}
\centering
\includegraphics[height=0.5in, keepaspectratio=true]{../Images/Roman_signature.jpg} \\
Roman Bulla, P. Eng. \\Power Systems Engineer \\
\includegraphics[height=0.5in, keepaspectratio=true]{../Images/Scott_signature.jpg} \\
Scott Vermeire \\Engineering Intern \\
\end{multicols}
%\end{comment}

%Vince Signature
\begin{comment}
\begin{multicols}{2}
\centering
\includegraphics[height=0.5in, keepaspectratio=true]{../Images/Roman_signature.jpg} \\
Roman Bulla, P. Eng. \\Power Systems Engineer \\
\includegraphics[height=0.5in, keepaspectratio=true]{../Images/Vince_signature.jpg} \\
Vince Klingenberger \\Electrical Engineering Technologist \\
\end{multicols}
\end{comment}

\pagebreak

%\subsubsection{Recommended Changes Summary}
%\label{af:results:afsr:sum}

%\noindent In order to achieve maximum selectivity and protection throughout your distribution system, we recommend the following changes be implemented: